This chapter introduces results that weren't in mathlib (or that were
in it but needed reformulation to stuck to our situation) and that are
mandatory for our goal.

\section{Group theory}
In this section, we fix $G$ group and $H$ a commutative sugroup of $G$.
We denote by $\mathcal{Z}_G$ the center of $G$ and by $e_G$ the neutral of $G$.

\subsection{Two lemmas about the center of a group}

\begin{proposition}
    \label{prop:center_mul_comm}
    \uses{}
    \lean{center_mul_comm}
    \leanok
    If $h$ is an element of the center of $G$, it commutes with every element of $G$.
\end{proposition}
    \begin{proof}
        \leanok
        Trivial, just a reformulation in terms of type instead of memebership.
    \end{proof}

\begin{proposition}
    \label{prop:center_mul_simp}
    \uses{center_mul_comm}
    \lean{center_mul}
    \leanok
    Let $h\in\mathcal{Z}_G$ such that $h=ab$ for some $(a,b)\in G^2$. Then, we also have
    $h=ba$.
\end{proposition}
\begin{proof}
    \leanok
    Suppose $h=ab$. Then $a=hb^{-1}=b^{-1}h$ because $h\in\mathcal{Z}_G$. Thus $ba=h$.
\end{proof}

\subsection{Quotient of $G$ by its center}

\begin{definition}
    \label{def:system_of_repr_center_set}
    \uses{}
    \lean{system_of_repr_center.set}
    \leanok
    We define the system of representatives of $G/\mathcal{Z}_G$ by picking up exactly one element
    in every classes. We denote it by $\mathcal{S}_{G/\mathcal{Z}_G}$ from now on and denote
    by $C_s$ the classe of $s$.
    \begin{proof}
        \leanok
        To do that in $L\exist\forall N$, we take the image of $G$ by the map $G \rightarrow G/\mathcal{Z}_G\to G$.
    \end{proof}
\end{definition}

\begin{proposition}
    \label{prop:system_of_repr_center_set_finite}
    \uses{def:system_of_repr_center_set}
    \lean{system_of_repr_center.finite}
    \leanok
    If $G$ is finite, then the system of representatives of $G/\mathcal{Z}_G$ is finite too.
\end{proposition}
\begin{proof}
    \leanok
    trivial
\end{proof}

\begin{proposition}
    \label{prop:system_of_repr_center_set_classes_disjoints}
    \uses{def:system_of_repr_center_set}
    \lean{system_of_repr_center.classes_disjoint}
    \leanok
    Given $g$ and $g'$ in the set of representatives of $G/\mathcal{Z}_G$, if $g\ne g'$
    then the classes of $g$ and $g'$ are disjoint.
\end{proposition}
\begin{proof}
    Add proof
\end{proof}

\begin{proposition}
    \label{prop:system_of_repr_center_set_union}
    \uses{def:system_of_repr_center_set}
    \lean{system_of_repr_center.classes_union_eq_univ}
    \leanok
    We have $\bigcup\limits_{s\in \mathcal{S}_{G/\mathcal{Z}_G}} C_s= G$.
\end{proposition}
\begin{proof}
    \leanok
    If $x$ is in the union, in particular it belongs to $G$. Let now $g$ be an element of $G$
    and let show that it belongs to one of the classes. We apply the map defines in \ref{def:system_of_repr_center_set}
    to $g$ and check that $g$ belongs to the class of this element.
\end{proof}

\begin{proposition}
    \label{def:system_of_repr_center_set_bij}
    \uses{def:system_of_repr_center_set}
    \lean{system_of_repr_center.set_bij}
    \leanok
    We have a bijection between $\mathcal{S}_{G/\mathcal{Z}_G}$ and $\{\bar{g} \in G/\mathcal{Z}_G}$
    given by the map $s\to \bar{s}$.
\end{proposition}
\begin{proof}
    \leanok
    We check it is a bijection.
\end{proof}

\begin{definition}
    \label{def:G_to_syst}
    \uses{system_of_repr_center_set}
    \lean{system_of_repr_center.G_to_syst}
    \leanok
    We define a map $\varphi_{G\mathcal{S}} : G \rightarrow \mathcal{S}_{G/\mathcal{Z}_G}$ that
    send every $g\in G$ to its representative.
\end{definition}

\begin{proposition}
    \label{prop:G_to_syst_simp}
    \uses{def:G_to_syst}
    \lean{system_of_repr_center.G_to_syst_simp}
    \leanok
    For every $g\in G$ and $h\in\mathbb{Z}_G$, we have $\varphi_{G\mathcal{S}}(gh)=\varphi_{G\mathcal{S}} (g)$.
\end{proposition}
\begin{proof}
    \leanok
    By definition $g*h$ belongs to the orbit of $g$, thus they have the same representative.
\end{proof}

\begin{definition}
    \label{def:G_to_center}
    \uses{system_of_repr_center_set}
    \lean{system_of_repr_center.G_to_center}
    \leanok
    We define a map $\psi_{G\mathcal{Z}_G} : G \rightarrow \mathcal{Z}_G$ that
    send every $g\in G$ to the corresponding $h\in \mathcal{Z}_G$ such that 
    $g=sh$ where $s$ is the representative of $g$.
\end{definition}

\begin{proposition}
    \label{prop:system_of_repr_center_G_to_center_G_to_syst_simp}
    \uses{prop:G_to_syst_simp,def:G_to_center}
    \lean{system_of_repr_center.G_eq_G_to_center_G_to_syst_simp}
    \leanok
     For every $g\in G$ the following identity holds : $g=\varphi_{G\mathcal{S}}(g) \psi_{G\mathcal{Z}_G}(g)$.
\end{proposition}
\begin{proof}
    \leanok
    By definition of $G/\mathcal{Z}_G$.
\end{proof}

\begin{proposition}
    \label{prop:system_of_repr_center_G_to_center_eq_G_G_to_syst_simp}
    \uses{prop:system_of_repr_center_G_to_center_G_to_syst_simp}
    \lean{system_of_repr_center.G_eq_G_to_center_G_to_syst_simp}
    \leanok
    For every $g\in G$ the following identity holds : $\psi_{G\mathcal{Z}_G}(g)=g\varphi_{G\mathcal{S}}(g)^{-1}$.
\end{proposition}
\begin{proof}
    \leanok
    Trivial with \ref{prop:system_of_repr_center_G_to_center_G_to_syst_simp}
\end{proof}

\begin{proposition}
    \label{prop:system_of_repr_center_G_to_center_syst_apply_simp}
    \uses{def:G_to_center}
    \lean{system_of_repr_center.G_to_center_syst_apply_simp}
    \leanok
    For every $g\in \mathcal{S}_{G/\mathcal{Z}_G}$, we have $\psi_{G\mathcal{Z}_G}(g)=e_G$.
\end{proposition}
\begin{proof}
    \leanok
    By definition of the map $\psi_{G\mathcal{Z}_G}$.
\end{proof}

\begin{proposition}
    \label{prop:system_of_repr_center_G_to_syst_simp_id}
    \uses{prop:system_of_repr_center_G_to_center_syst_apply_simp,prop:system_of_repr_center_G_to_center_eq_G_G_to_syst_simp}
    \lean{system_of_repr_center.G_to_syst_simp_id}
    \leanok
    For every $g\in \mathcal{S}_{G/\mathcal{Z}_G}$, we have $\varphi_{G\mathcal{S}}(g)=g$.
\end{proposition}
\begin{proof}
    \leanok
    We have $g=\varphi_{G\mathcal{S}}(g) \psi_{G\mathcal{Z}_G}(g)$ by \ref{prop:system_of_repr_center_G_to_center_G_to_syst_simp}.
    But $\psi_{G\mathcal{Z}_G}(g)=e_G$ by \ref{prop:system_of_repr_center_G_to_center_syst_apply_simp}.
    Thus we get the result.
\end{proof}

\begin{proposition}
    \label{prop:system_of_repr_center_G_to_center_mul_simp}
    \uses{prop:system_of_repr_center_G_to_center_eq_G_G_to_syst_simp}
    \lean{system_of_repr_center.G_to_center_mul_simp}
    \leanok
    For every $g\in G$ and $h\in\mathcal{Z}_G$, we have $\psi_{G\mathcal{Z}_G}(gh)=h\psi_{G\mathcal{Z}_G}(g)$.
\end{proposition}
\begin{proof}
    \leanok
    With \label{prop:system_of_repr_center_G_to_center_eq_G_G_to_syst_simp} we have $\psi_{G\mathcal{Z}_G}(g)=g\varphi_{G\mathcal{S}}(g)^{-1}$
    and $\psi_{G\mathcal{Z}_G}(gh)=gh\varphi_{G\mathcal{S}}(gh)^{-1}$.
    But with \ref{prop:G_to_syst_simp}, we have $\varphi_{G\mathcal{S}}(gh)=\varphi_{G\mathcal{S}} (g)$.
    Thus, $\psi_{G\mathcal{Z}_G}(gh)=gh\varphi_{G\mathcal{S}}(gh)^{-1}=gh\varphi_{G\mathcal{S}} (g)^{-1}=
    hg\varphi_{G\mathcal{S}} (g)^{-1}=h\psi_{G\mathcal{Z}_G}(g)$ withe the first equality.
\end{proof}

\begin{definition}
    \label{def:system_of_repr_set_center_iso_G}
    \uses{prop:system_of_repr_center_G_to_center_mul_simp,prop:system_of_repr_center_G_to_syst_simp_id}
    \lean{system_of_repr_center.set_center_iso_G}
    \leanok
    We define a bijection from $G$ to $\mathcal{Z}_G\times\mathcal{S}_{G/\mathcal{Z}_G}$ by 
    $g\mapsto (\psi_{G\mathcal{Z}_G}(g),\varphi_{G\mathcal{S}} (g))$.
    \begin{proof}
        \leanok
        We check the axioms of a bijection.
    \end{proof}
\end{definition}

\begin{definition}
    \label{system_of_repr_set_center_iso_G_sigma}
    \uses{def:system_of_repr_center_set_center_iso_G}
    \lean{system_of_repr_center.set_center_iso_G_sigma}
    \leanok
    Bijection \ref{def:set_center_iso_G} empacked as Sigma type instead of cartesian product.
\end{definition}

\begin{proposition}
    \label{prop:system_of_repr_center_set_center_eq_G }
    \uses{prop:system_of_repr_center_G_to_center_eq_G_G_to_syst_simp}
    \lean{system_of_repr_center.set_center_eq_G}
    \leanok
    We have $G = \{gh,\ g\in \mathcal{S}_{G/\mathcal{Z}_G},\ h\in \mathcal{Z}_G\}$.
\end{proposition}
\begin{proof}
    \leanok
    The inclusion $\{gh,\ g\in \mathcal{S}_{G/\mathcal{Z}_G},\ h\in \mathcal{Z}_G\}$ is trivial.
    The converse is given by \ref{prop:system_of_repr_center_G_to_center_eq_G_G_to_syst_simp}.
\end{proof}


\section{Direct sums and tensor products}

\subsection{Direct sums}

\begin{definition}
    \label{def:DirectSum_equiv}
    \uses{}
    \lean{DirectSum_equiv}
    \leanok
    If we have two families $(\beta_i)_{i\in I}$ and $(\gamma_i)_{i\in I}$ of additive commutative
    monoids such that for every $i\in I$, we have an additive bijection $\varphi_i$ between 
    $\beta_i$ and $\gamma_i$, then we have an additive bijection
    between $\bigoplus\limits_{i\in I}\beta_i$ and $\bigoplus\limits_{i\in I}\gamma_i$.
    \begin{proof}
        \leanok
        We send $\sum\limits_{i\in I}x_i$ on $\sum\limits_{i\in I}\varphi(x_i)$ and we check
        that it's an additive bijection. 
    \end{proof}
\end{definition}

\begin{definition}
    \label{def:DirectSum_equiv_linearmap}
    \uses{def:DirectSum_equiv}
    \lean{DirectSum_equiv_linearmap}
    \leanok
    Let $A$ be a semiring. If we have two families $(\beta_i)_{i\in I}$ and $(\gamma_i)_{i\in I}$ of additive commutative
    monoids such that for every $i\in I$, $\beta_i$ and $\gamma_i$ are $A-$module and we have a $A-$linear bijection $\varphi_i$ between 
    $\beta_i$ and $\gamma_i$, then we have a $A$ linear bijection
    between $\bigoplus\limits_{i\in I}\beta_i$ and $\bigoplus\limits_{i\in I}\gamma_i$.
    \begin{proof}
        \leanok
        We take the map defined in \ref{def:DirectSum_equiv} which became $A-$linear by
        the new properties of the $\beta_i$ and $\gamma_i$.
    \end{proof}
\end{definition}

\begin{proposition}
    \label{prop:DirectSum_eq_sum_direct}
    \uses{}
    \lean{DirectSum_eq_sum_direct}
    \leanok
    Let $I$ be a finite set and $(\beta_i)_{i\in I}$ a family of additive commutative monoids.
    Let $\Phi$ be the natural map sending $\beta_{i_0}$ to $\bigoplus\limits_{i\in I}\beta_i$.
    Then, for every $x:=(x_i)_{i\in I}$ such that $x_i\in\beta_i$ for all $i\in I$, then for every $j\in I$,
    the following equality holds : $\left(\sum\limits_{i\in I} \Phi(x_i)\right)_j=x_j$.
\end{proposition}
\begin{proof}
    \leanok
    We obvioulsy have $\left(\sum\limits_{i\in I} \Phi(x_i)\right)=\sum\limits_{i\in I}x_i$,
    which immediately gives the result.
\end{proof}

\subsection{Tensor products}

\begin{definition}
    \label{def:iso_hom_tens}
    \uses{}
    \lean{iso_hom_tens}
    \leanok
    Let $A$ be ring, $B$ an $A-$algebra, $M$ an $A-$module and $N$ a $B-$module. Then,
    $\text{Hom}_B((B\otimes_AM),N)\cong \text{Hom}_A(M,N)$.
    \begin{proof}
        \leanok
        We consider the map sending $\varphi\in\text{Hom}_B((B\otimes_AM),N)$ to the $A-$linear map
        $\Phi_\varphi : M \rightarrow N$ defined by $\Phi_\varphi(x)=\varphi (1\otimes_A x)$ for every $x\in M$.
        It is injective : if $\Phi_{\varphi_1}=\Phi_{\varphi_2}$, then $\varphi_1(1\otimes_A x)=\varphi_2(1\otimes_A x)$
        for every $x\in M$. Thus $\varphi_1=\varphi_2$ by $B-$linearity.
        It is surjective : let $\varphi$ be a $A$-linear map from $M$ to $N$. Let consider the $B-$linear map
        $\psi$ from $B\otimes_AM$ to $N$ define by $\psi (b\otimes_A m) = b \varphi (m)$. We have then 
        $\Psi_\psi(x)=\psi (1\otimes_A x)= \varphi(x)$.

    \end{proof}
\end{definition}


\section{Monoid algebra}
A lot of the results in this section wouldn't really appear in classical mathematics papers, but they
are needed to ensure that LEAN understand the operations we will do later.
\newline

From now on, $\mathbb{K}$ is a field, $G$ is a group and $H$ is a subgroup of $G$.
We define $mathcal{Z}_G$ as the center of $G$.

\subsection{Setting up operations, coercions and instances in LEAN}

\begin{definition}
    \label{def:Map_kHkG}
    \uses{}
    \lean{Map_KHKG}
    \leanok
    Given $\mathbb{K}$ a field, $G$ a group and $H$ a subgroup of $G$, 
    we have a trivial ring homomorphism $\varphi_{kHkG}$ from $\mathbb{K}[H]$ to $\mathbb{K}[G]$.
    \begin{proof}
        \leanok
        Trivial.
    \end{proof}
\end{definition}

\begin{proposition}
    \label{prop:Map_kHkG_inj}
    \uses{def:Map_kHkG}
    \lean{Map_kHkG_inj}
    \leanok
    The map defined in \ref{def:Map_kHkG} is injective.
\end{proposition}
\begin{proof}
    \leanok
    Trivial.
\end{proof}

\begin{proposition}
    \label{prop:Map_kHkG_single_apply}
    \uses{def:Map_kHkG}
    \lean{Map_kHkG_single_apply}
    \leanok
    We have an equality between $h\in H$ seen as an element of $\mathbb{K}[H]$ and $h$ (seen as an element 
    of $G$) seens as an element of $\mathbb{K}[G]$.
\end{proposition}
\begin{proof}
    \leanok
    Some LEAN stuff.
\end{proof}

\begin{proposition}
    \label{prop:Map_kHkG_k_linear}
    \uses{def:Map_kHkG}
    \lean{Map_kHkG_k_linear}
    \leanok
    The map defined in \ref{def:Map_kHkG} is $k$ linear.
\end{proposition}
\begin{proof}
    \leanok
    Trivial.
\end{proof}

\begin{definition}
    \label{def:Coe_kH_kG}
    \uses{def:Map_kHkG}
    \lean{Coe_kH_kG}
    \leanok
    We define a coercion from elements of $\mathbb{K}[H]$ to $\mathbb{K}[G]$ by the map defined in \ref{def:Map_kHkG}.
    \begin{proof}
        \leanok
        Some LEAN stuff.
    \end{proof}
\end{definition}

\begin{definition}
    \label{def:Set_Coe}
    \uses{def:Map_kHkG}
    \lean{Set_Coe}
    \leanok
    We define a coercion from sets of elements of $\mathbb{K}[H]$ to sets of elements of $\mathbb{K}[G]$ by 
    the map defined in \ref{def:Map_kHkG}.
    \begin{proof}
        \leanok
        Some LEAN stuff.
    \end{proof}
\end{definition}

\begin{definition}
    \label{def:Smul_kHkG}
    \uses{def:Map_kHkG,def:Coe_kH_kG}
    \lean{Smul_kHkG}
    \leanok
    We define a multiplication between elements of $\mathbb{K}[H]$ and elements of $\mathbb{K}[G]$ by 
    $kH*kG=\varphi_{kHkG}(kH)\times kG$.
    \begin{proof}
        \leanok
        Some LEAN stuff.
    \end{proof}
\end{definition}

\begin{proposition}
    \label{prop:kG_is_kCenter_Algebra}
    \uses{def:Map_kHkG,def:Smul_kHkG}
    \lean{kG_is_kCenter_Algebra }
    \leanok
    $\mathbb{K}[G]$ is a $\mathbb{K}[\mathcal{Z}_G]$ algebra.
\end{proposition}
\begin{proof}
    \leanok
    We check the axiom of an algebra.
\end{proof}

\begin{proposition}
    \label{prop:kG_is_kH_Algebra}
    \uses{prop:kG_is_kCenter_Algebra}
    \lean{kG_is_kH_Algebra}
    \leanok
    If there exists a morphism from $H$ to $\mathcal{Z}_G$, then $\mathbb{K}[G]$ is a $\mathbb{K}[H]$ algebra.
\end{proposition}
\begin{proof}
    \leanok
    We check the axiom of an algebra with the action of $\mathbb{K}[H]$ on $\mathbb{K}[G]$ given by the morphism.
\end{proof}

\begin{proposition}
    \label{prop:center_commutes_single}
    \uses{def:Coe_kH_kG}
    \lean{center_commutes_single}
    \leanok
    Let $x\in\mathbb{K}[\mathcal{Z}_G]$ and $g\in G$. Then $g\times x = x \times g$.
\end{proposition}
\begin{proof}
    \leanok
    We have $g\times x = g\times \sum\limits_{h\in\mathcal{Z}_g}a_hh=\sum\limits_{h\in\mathcal{Z}_g}a_hgh
    =\sum\limits_{h\in\mathcal{Z}_g}a_hhg=\left(\sum\limits_{h\in\mathcal{Z}_g}a_hgh\right)\times g$.
\end{proof}

\begin{definition}
    \label{def:center_sub_module}
    \uses{def:Map_kHkG}
    \lean{center_sub_module}
    \leanok
    $\mathbb{K}[\mathcal{Z}_G]$ defines a $\mathbb{K}[G]$ submodule.
    \begin{proof}    
        \leanok
        We check the axioms.
    \end{proof}
\end{definition}

\begin{definition}
    \label{def:hmul_g_kH_kG}
    \uses{def:Map_kHkG}
    \lean{hmul_g_kH_kG}
    \leanok
    We define the multiplciation of elements $g\in G$ and $kH\in\mathbb{K}[\mathcal{Z}_G]$ in $\mathbb{K}[G]$
    by $g\times kH=g\times \varphi_{kHkG}$
    \begin{proof}    
        \leanok
        Some LEAN stuff.
    \end{proof}
\end{definition}

\begin{definition}
    \label{def:hmul_g_kG}
    \uses{def:Map_kHkG}
    \lean{hmul_g_kG}
    \leanok
    We define the multiplciation of elements $g\in G$ and $kG\in\mathbb{K}[G]$ in $\mathbb{K}[G]$
    by $g\times kG$.
    \begin{proof}    
        \leanok
        Some LEAN stuff.
    \end{proof}
\end{definition}

\begin{proposition}
    \label{prop:hmul_g_kH_kG_distrib}
    \uses{def:hmul_g_kH_kG}
    \lean{hmul_g_kH_kG_distrib}
    \leanok
    Elements of $G$ are distributive over $\mathbb{K}[\mathcal{Z}_G]$
\end{proposition}
\begin{proof}
    \leanok
    Trivial, LEAN stuff.
\end{proof}

\subsection{Splitting of $\mathbb{K}[G]$ as a direct sum}

We use the notation of the section 1 concerning $G/\mathcal{Z}_G$
The main goal of this part is to formalize the following result :
$\mathbb{K}[G] \cong \bigoplus\limits_{g\in \mathcal{S}_{G/\mathcal{Z}_G} } g \mathbb{K}[\mathcal{Z}_G]$.

\begin{definition}
    \label{def:gkH_map}
    \uses{}
    \lean{gkH_map}
    \leanok
    Let $g\in G$ be fixed. The morphism $\varhpi_g : \mathcal{Z}_G\rightarrow G$ defined by
    $\varphi_G(x)=gx$ induced a $\mathbb{K}$-linear map $\Gamma$ from $\mathbb{K}[\mathcal{Z}_G]$ to $\mathbb{K}[G]$.
    \begin{proof}    
        \leanok
        Trivial.
    \end{proof}
\end{definition}

\begin{proposition}
    \label{prop:gkH_map_eq}
    \uses{def:gkH_map,def:hmul_g_kH_kG}
    \lean{gkH_map_eq}
    \leanok
    For all $x\in\mathbb{K}[\mathcal{Z}[G]]$, we have $\Gamma(x)=g\times x$
\end{proposition}
\begin{proof}
    \leanok
    LEAN stuff to setup a simp lemma.
\end{proof}