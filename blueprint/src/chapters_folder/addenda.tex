This chapter introduces results that weren't in mathlib (or that were
in it but needed reformulation to stuck to our situation) and that are
mandatory for our goal.

\section{Group theory}
In this section, we fix $G$ group and $H$ a commutative sugroup of $G$.
We denote by $\mathcal{Z}_G$ the center of $G$ and by $e_G$ the neutral of $G$.

\subsection{Two lemmas about the center of a group}

\begin{proposition}
    \label{prop:center_mul_comm}
    \lean{center_mul_comm}
    \leanok
    If $h$ is an element of the center of $G$, it commutes with every element of $G$.
\end{proposition}
\begin{proof}
    \leanok
    Trivial, just a reformulation in terms of type instead of memebership, useful for the $LEAN$ part.
\end{proof}

\begin{proposition}
    \label{prop:center_mul_simp}
    \uses{prop:center_mul_comm}
    \lean{center_mul}
    \leanok
    Let $h\in\mathcal{Z}_G$ such that $h=ab$ for some $(a,b)\in G^2$. Then, we also have
    $h=ba$.
\end{proposition}
\begin{proof}
    \leanok
    Suppose $h=ab$. Then $a=hb^{-1}=b^{-1}h$ because $h\in\mathcal{Z}_G$. Thus $ba=h$.
\end{proof}

