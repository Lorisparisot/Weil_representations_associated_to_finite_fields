This chapter introduces results that weren't in mathlib (or that were
in it but needed reformulation to stuck to our situation) and that are
mandatory for our goal.

\section{Group theory}
In this section, we fix $G$ group and $H$ a commutative sugroup of $G$.
We denote by $\mathcal{Z}_G$ the center of $G$ and by $e_G$ the neutral of $G$.

\subsection{Two lemmas about the center of a group}

\begin{proposition}
    \label{prop:center_mul_comm}
    \lean{center_mul_comm}
    \leanok
    If $h$ is an element of the center of $G$, it commutes with every element of $G$.
\end{proposition}
\begin{proof}
    \leanok
    Trivial, just a reformulation in terms of type instead of memebership, useful for the $LEAN$ part.
\end{proof}

\begin{proposition}
    \label{prop:center_mul_simp}
    \uses{prop:center_mul_comm}
    \lean{center_mul}
    \leanok
    Let $h\in\mathcal{Z}_G$ such that $h=ab$ for some $(a,b)\in G^2$. Then, we also have
    $h=ba$.
\end{proposition}
\begin{proof}
    \leanok
    Suppose $h=ab$. Then $a=hb^{-1}=b^{-1}h$ because $h\in\mathcal{Z}_G$. Thus $ba=h$.
\end{proof}

\subsection{Quotient of a group by its center}

\begin{definition}[Representatives system]
    \label{def:system_of_repr_center_set}
    \lean{system_of_repr_center.set}
    \leanok
    We define the system of representatives of $G/\mathcal{Z}_G$ by picking up exactly one element
    in every classes. We denote it by $\mathcal{S}_{G/\mathcal{Z}_G}$ from now on and denote
    by $C_s$ the classe of $s$.
    \begin{proof}
        \leanok
        To do that in $LEAN$, we take the image of $G$ by the map $G \rightarrow G/\mathcal{Z}_G\to G$.
    \end{proof}
\end{definition}

\begin{proposition}
    \label{prop:system_of_repr_center_set_finite}
    \uses{def:system_of_repr_center_set}
    \lean{system_of_repr_center.finite}
    \leanok
    If $G$ is finite, then the system of representatives of $G/\mathcal{Z}_G$ is finite too.
\end{proposition}
\begin{proof}
    \leanok
    trivial
\end{proof}

\begin{proposition}
    \label{prop:system_of_repr_center_set_classes_disjoints}
    \uses{def:system_of_repr_center_set}
    \lean{system_of_repr_center.classes_disjoint}
    \leanok
    Given $g$ and $g'$ in the set of representatives of $G/\mathcal{Z}_G$, if $g\ne g'$
    then the classes of $g$ and $g'$ are disjoint.
\end{proposition}
\begin{proof}
    \leanok
    Suppose that the classes aren't disjoints. Then there exists $y$
    such that $y\sim g$ and $y\sim g'$. Thus $g \sim g'$ and their classes
    are equal. But $g$ and $g'$ belongs to the set of representatives.
    Thus $g=g'$.
\end{proof}

\begin{proposition}
    \label{prop:system_of_repr_center_set_union}
    \uses{def:system_of_repr_center_set}
    \lean{system_of_repr_center.classes_union_eq_univ}
    \leanok
    We have $\bigcup\limits_{s\in \mathcal{S}_{G/\mathcal{Z}_G}} C_s= G$.
\end{proposition}
\begin{proof}
    \leanok
    If $x$ is in the union, in particular it belongs to $G$. Let now $g$ be an element of $G$
    and let show that it belongs to one of the classes. We apply the map defines in \ref{def:system_of_repr_center_set}
    to $g$ and check that $g$ belongs to the class of this element.
\end{proof}

\begin{proposition}
    \label{def:system_of_repr_center_set_bij}
    \uses{def:system_of_repr_center_set}
    \lean{system_of_repr_center.set_bij}
    \leanok
    We have a bijection between $\mathcal{S}_{G/\mathcal{Z}_G}$ and $\{\bar{g} \in G/\mathcal{Z}_G\}$
    given by the map $s\to \bar{s}$.
\end{proposition}
\begin{proof}
    \leanok
    We check it is a bijection.
\end{proof}

\begin{definition}
    \label{def:G_to_syst}
    \uses{def:system_of_repr_center_set}
    \lean{system_of_repr_center.G_to_syst}
    \leanok
    We define a map $\varphi_{G\mathcal{S}} : G \rightarrow \mathcal{S}_{G/\mathcal{Z}_G}$ that
    send every $g\in G$ to its representative.
\end{definition}

\begin{proposition}
    \label{prop:G_to_syst_simp}
    \uses{def:G_to_syst}
    \lean{system_of_repr_center.G_to_syst_simp}
    \leanok
    For every $g\in G$ and $h\in\mathbb{Z}_G$, we have $\varphi_{G\mathcal{S}}(gh)=\varphi_{G\mathcal{S}} (g)$.
\end{proposition}
\begin{proof}
    \leanok
    By definition $g*h$ belongs to the orbit of $g$, thus they have the same representative.
\end{proof}

\begin{definition}
    \label{def:G_to_center}
    \uses{def:system_of_repr_center_set}
    \lean{system_of_repr_center.G_to_center}
    \leanok
    We define a map $\psi_{G\mathcal{Z}_G} : G \rightarrow \mathcal{Z}_G$ that
    send every $g\in G$ to the corresponding $h\in \mathcal{Z}_G$ such that 
    $g=sh$ where $s$ is the representative of $g$.
\end{definition}

\begin{proposition}
    \label{prop:system_of_repr_center_G_to_center_G_to_syst_simp}
    \uses{prop:G_to_syst_simp,def:G_to_center}
    \lean{system_of_repr_center.G_eq_G_to_center_G_to_syst_simp}
    \leanok
     For every $g\in G$ the following identity holds : $g=\varphi_{G\mathcal{S}}(g) \psi_{G\mathcal{Z}_G}(g)$.
\end{proposition}
\begin{proof}
    \leanok
    By definition of $G/\mathcal{Z}_G$.
\end{proof}

\begin{proposition}
    \label{prop:system_of_repr_center_G_to_center_eq_G_G_to_syst_simp}
    \uses{prop:system_of_repr_center_G_to_center_G_to_syst_simp}
    \lean{system_of_repr_center.G_to_center_eq_G_G_to_syst_simp}
    \leanok
    For every $g\in G$ the following identity holds : $\psi_{G\mathcal{Z}_G}(g)=g\varphi_{G\mathcal{S}}(g)^{-1}$.
\end{proposition}
\begin{proof}
    \leanok
    Trivial with \ref{prop:system_of_repr_center_G_to_center_G_to_syst_simp}
\end{proof}

\begin{proposition}
    \label{prop:system_of_repr_center_G_to_center_syst_apply_simp}
    \uses{def:G_to_center}
    \lean{system_of_repr_center.G_to_center_syst_apply_simp}
    \leanok
    For every $g\in \mathcal{S}_{G/\mathcal{Z}_G}$, we have $\psi_{G\mathcal{Z}_G}(g)=e_G$.
\end{proposition}
\begin{proof}
    \leanok
    By definition of the map $\psi_{G\mathcal{Z}_G}$.
\end{proof}

\begin{proposition}
    \label{prop:system_of_repr_center_G_to_syst_simp_id}
    \uses{prop:system_of_repr_center_G_to_center_syst_apply_simp,prop:system_of_repr_center_G_to_center_eq_G_G_to_syst_simp}
    \lean{system_of_repr_center.G_to_syst_simp_id}
    \leanok
    For every $g\in \mathcal{S}_{G/\mathcal{Z}_G}$, we have $\varphi_{G\mathcal{S}}(g)=g$.
\end{proposition}
\begin{proof}
    \leanok
    We have $g=\varphi_{G\mathcal{S}}(g) \psi_{G\mathcal{Z}_G}(g)$ by \ref{prop:system_of_repr_center_G_to_center_G_to_syst_simp}.
    But $\psi_{G\mathcal{Z}_G}(g)=e_G$ by \ref{prop:system_of_repr_center_G_to_center_syst_apply_simp}.
    Thus we get the result.
\end{proof}

\begin{proposition}
    \label{prop:system_of_repr_center_G_to_center_mul_simp}
    \uses{prop:system_of_repr_center_G_to_center_eq_G_G_to_syst_simp}
    \lean{system_of_repr_center.G_to_center_mul_simp}
    \leanok
    For every $g\in G$ and $h\in\mathcal{Z}_G$, we have $\psi_{G\mathcal{Z}_G}(gh)=h\psi_{G\mathcal{Z}_G}(g)$.
\end{proposition}
\begin{proof}
    \leanok
    With \label{prop:system_of_repr_center_G_to_center_eq_G_G_to_syst_simp} we have $\psi_{G\mathcal{Z}_G}(g)=g\varphi_{G\mathcal{S}}(g)^{-1}$
    and $\psi_{G\mathcal{Z}_G}(gh)=gh\varphi_{G\mathcal{S}}(gh)^{-1}$.
    But with \ref{prop:G_to_syst_simp}, we have $\varphi_{G\mathcal{S}}(gh)=\varphi_{G\mathcal{S}} (g)$.
    Thus, $\psi_{G\mathcal{Z}_G}(gh)=gh\varphi_{G\mathcal{S}}(gh)^{-1}=gh\varphi_{G\mathcal{S}} (g)^{-1}=
    hg\varphi_{G\mathcal{S}} (g)^{-1}=h\psi_{G\mathcal{Z}_G}(g)$ withe the first equality.
\end{proof}

\begin{definition}
    \label{def:system_of_repr_center_set_center_iso_G}
    \uses{prop:system_of_repr_center_G_to_center_mul_simp,prop:system_of_repr_center_G_to_syst_simp_id}
    \lean{system_of_repr_center.set_center_iso_G}
    \leanok
    We define a bijection from $G$ to $\mathcal{Z}_G\times\mathcal{S}_{G/\mathcal{Z}_G}$ by 
    $g\mapsto (\psi_{G\mathcal{Z}_G}(g),\varphi_{G\mathcal{S}} (g))$.
    \begin{proof}
        \leanok
        We check the axioms of a bijection.
    \end{proof}
\end{definition}

\begin{definition}
    \label{system_of_repr_set_center_iso_G_sigma}
    \uses{def:system_of_repr_center_set_center_iso_G}
    \lean{system_of_repr_center.set_center_iso_G_sigma}
    \leanok
    The bijection \ref{def:system_of_repr_center_set_center_iso_G} empacked as Sigma type instead of cartesian product.
    Useful for $LEAN$.
\end{definition}

\begin{proposition}
    \label{prop:system_of_repr_center_set_center_eq_G }
    \uses{prop:system_of_repr_center_G_to_center_eq_G_G_to_syst_simp}
    \lean{system_of_repr_center.set_center_eq_G}
    \leanok
    We have $G = \{gh,\ g\in \mathcal{S}_{G/\mathcal{Z}_G},\ h\in \mathcal{Z}_G\}$.
\end{proposition}
\begin{proof}
    \leanok
    The inclusion $\{gh,\ g\in \mathcal{S}_{G/\mathcal{Z}_G},\ h\in \mathcal{Z}_G\}$ is trivial.
    The converse is given by \ref{prop:system_of_repr_center_G_to_center_eq_G_G_to_syst_simp}.
\end{proof}


