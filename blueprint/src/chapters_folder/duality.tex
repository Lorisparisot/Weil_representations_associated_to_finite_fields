L'article fait le choix d'une identification non-canonique du bidual :
$<x,y>=-<y,x>$. Certaines propriétés dépendent exclusivement de cette
convention \textit{(voir par exemple)}.

\section{Mise en place des conventions}

\begin{definition}[Forme sur $V^{**}\times V$]
    \label{def:form_dual}
    \uses{}
    \lean{convention_dual}
    \leanok 

    On définit une application bilinéaire de $V^{**}\times V$ dans $k$ par 
    $(x,y)\mapsto - y(x)$.
    \begin{proof}
        \leanok
        On vérifie la bilinéarité.
    \end{proof}
\end{definition}
On accompage également ce lemme d'un lemme simp.

\begin{definition}[Application $V$ dans $V^{**}$]
    \label{def:v_to_dual_dual}
    \uses{}
    \lean{convention_eval}
    \leanok 

    On définit une application linéaire de $V$ dans $V^{**}$ par 
    $v\mapsto (\varphi\mapsto -\varphi(v))$.
    \begin{proof}
        \leanok
        On vérifie que c'est une application linéaire
    \end{proof}
\end{definition}
On accompagne également ce lemme d'un lemme simp.

\begin{proposition}[Application $V$ dans $V^{**}$]
    \label{def:identification_dual}
    \uses{def:v_to_dual_dual}
    \lean{convention_eval_iso}
    \leanok 

    Si $V$ est réflexif, alors l'application définie en \ref{def:v_to_dual_dual} est une 
    bijection linéaire de $V$ dans $V^{**}$.
    \begin{proof}
        \leanok
        On vérifie que c'est une bijection en donnant un inverse.
    \end{proof}
\end{proposition}
On accompagne également ce lemme d'un lemme simp.

\section{Quelques résultats sur la forme commutateur}

\begin{definition}[Forme commutateur]
    \label{def:form_com}
    \uses{}
    \lean{form_commutator}
    \leanok 

    Si $V$ est réflexif, l'application $((x_1,y_1),(x_2,y_2))\mapsto y_1(x_2)-y_2(x_1)$ est une forme bilinéaire sur 
    $V\times V^*$.
    \begin{proof}
        \leanok
        On vérifie la bilinéarité.
    \end{proof}
\end{definition}

\begin{proposition}[Non dégénéréscence]
    \label{def:non_dege_form_com}
    \uses{def:form_com}
    \lean{form_commutator_non_degenerate}
    \leanok 

    La forme définie en \ref{def:form_com} est non dégénérée.
    \begin{proof}
        \leanok
        Supposons que la forme soit dégénérée. Alors il existe $h:=(x,y)\in V\times V^*$ tel
        que $h\ne 0$ et $y(x')-y'(x)=0$ pour tout $(x',y')\in V\times V^*$.
        En particulier, pour $y'=0$, $y(x')=0$ pour tout $x'$ et donc $y=0$.
        Pour $x'=0$, $y'(x)=0$ pour tout $y'$ et donc $x=0$.
        Le seul vecteur correspondant est $0$, ce qui contredit l'hypothèse.
    \end{proof}
\end{proposition}