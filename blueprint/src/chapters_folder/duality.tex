The article chooses to have a non canonical identification between $V$ and its bidual :
$<x,y>=-<y,x>$. Some properties only rely on this identification
 \textit{(see \ref{prop:antiisoH} for an example)}.

\section{Setting up the conventions}

\begin{definition}[Bilinear form on $V^{**}\times V$]
    \label{def:form_dual}
    \uses{}
    \lean{convention_dual}
    \leanok 

    We define a bilinear form on $V^{**}\times V$ by 
    $(x,y)\mapsto - y(x)$.
    \begin{proof}
        \leanok
        We check the bilinearity.
    \end{proof}
\end{definition}
We set up also a simp lemma for the evaluation of the bilinear form.

\begin{definition}[Map from $V$ to $V^{**}$]
    \label{def:v_to_dual_dual}
    \uses{}
    \lean{convention_eval}
    \leanok 

    We define a linear map from $V$ to $V^{**}$ by 
    $v\mapsto (\varphi\mapsto -\varphi(v))$.
    \begin{proof}
        \leanok
        We check the linearity of the map.
    \end{proof}
\end{definition}
We set up also a simp lemma for the evaluation of the map.

\begin{proposition}[Bijective map from $V$ to $V^{**}$]
    \label{def:identification_dual}
    \uses{def:v_to_dual_dual}
    \lean{convention_eval_iso}
    \leanok 

    If $V$ is reflexive, then the linear map defined in \ref{def:v_to_dual_dual} is
    a bijective linear map from $V$ to $V^{**}$.
\end{proposition}
\begin{proof}
    \leanok
    We check it's a bijective map by giving the explicit inverse map.
\end{proof}
We set up also a simp lemma for the evaluation of the bijective map.

\section{Some results about the commutator bilinear form}

\begin{definition}[Commutator form]
    \label{def:form_com}
    \uses{}
    \lean{form_commutator}
    \leanok 

    If $V$ is reflexive, the map $((x_1,y_1),(x_2,y_2))\mapsto y_1(x_2)-y_2(x_1)$ is a bilinear form on 
    $V\times V^*$.
    \begin{proof}
        \leanok
        We check the bilinearity.
    \end{proof}
\end{definition}

\begin{proposition}[Nondegeneracy]
    \label{def:non_dege_form_com}
    \uses{def:form_com}
    \lean{form_commutator_non_degenerate}
    \leanok 

    The bilinear form defined in \ref{def:form_com} is a nondegeneracy bilinear form.
\end{proposition}
\begin{proof}
    \leanok
    Suppose it is degeneracy. Then there exists $h:=(x,y)\in V\times V^*$ such that
    $h\ne 0$ and $y(x')-y'(x)=0$ for all $(x',y')\in V\times V^*$.
    In particular, for $y'=0$, $y(x')=0$ for all $x'$, so $y=0$.
    Then, for $x'=0$, $y'(x)=0$ for all $y'$, so $x=0$.
    Thus $h=0$. Contradiction.
\end{proof}