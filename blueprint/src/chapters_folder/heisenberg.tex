\begin{definition}[Ensemble d'Heisenberg]
    \label{def:heisenberg}
    \lean{Heisenberg}
    \leanok

    Etant donné $k$ un corps, $V$ un espace vectoriel de dimension finie sur
    $k$ et $V^*$ le dual de $V$, on définit l'ensemble d'Heisenberg par 
    $\mathcal{H}(V):=\{(z,x,y) \in k\times V\times V^*\}$.
\end{definition}

\begin{definition}[Multiplication sur l'ensemble d'Heisenberg]
    \label{def:mul_H}
    \uses{def:heisenberg}
    \lean{Heisenberg.mul}
    \leanok 

    On définit une loi interne sur l'ensemble d'Heisenberg par :
    $(z_1,x_1,y_1)*(z_2,x_2,y_2) = (z_1+z_2+y_1(x_2),x_1+x_2,y_1+y_2)$ pour
    tout $(z_1,x_1,y_1),(z_2,x_2,y_2)\in\mathcal{H}(V)$.
\end{definition}

\begin{definition}[Inverse d'un élément d'Heisenberg]
    \label{def:inv_H}
    \uses{def:heisenberg}
    \lean{Heisenberg.inverse}
    \leanok 

    L'inverse d'un élément $(z,x,y)\in\mathcal{H}(V)$ est donné par 
    $(-z- y(-x), - x ,- y)$.
\end{definition}

\begin{proposition}[Groupe d'Heisenberg]
    \label{prop:group_H}
    \uses{def:heisenberg,def:mul_H,def:inv_H}
    \lean{Heisenberg.group}
    \leanok
    Muni de la loi de composition interne, Heisenberg est un groupe.
    \begin{proof}
        On vérifie les axiomes et on fait les calculs. Pas de difficultés
        notoires.
    \end{proof}
\end{proposition}

\begin{definition}[Centre d'Heisenberg]
    \label{def:center_H}
    \uses{def:heisenberg}
    \lean{Heisenberg.center}
    \leanok 

    On définit le centre d'Heisenberg comme étant l'ensemble des éléments
    $\mathcal{Z}_{\mathcal{H}(V)}:=\{(z,0,0)\in\mathcal{H}(V),\ z\in k\}$
\end{definition}

\begin{proposition}[Centre est un sous-groupe]
    \label{prop:center_H_subgroup}
    \uses{def:center_H, prop:group_H}
    \lean{Heisenberg.center_is_subgroup}
    \leanok

    Le centre d'Heisenberg $\mathcal{Z}_{\mathcal{H}(V)}$ est un sous-groupe
    de $\mathcal{H}(V)$.
    \begin{proof}
    On vérifie les axiomes et on fait les calculs. Pas de difficultés notoires.
    \end{proof}
\end{proposition}
