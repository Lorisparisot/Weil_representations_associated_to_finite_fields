\section{Construction}

\begin{definition}[Ensemble d'Heisenberg]
    \label{def:heisenberg}
    \lean{Heisenberg}
    \leanok

    Etant donné $k$ un corps, $V$ un espace vectoriel de dimension finie sur
    $k$ et $V^*$ le dual de $V$, on définit l'ensemble d'Heisenberg par 
    $\mathcal{H}(V):=\{(z,x,y) \in k\times V\times V^*\}$.
\end{definition}

\begin{definition}[Multiplication sur l'ensemble d'Heisenberg]
    \label{def:mul_H}
    \uses{def:heisenberg}
    \lean{Heisenberg.mul}
    \leanok 

    On définit une loi interne sur l'ensemble d'Heisenberg par :
    $(z_1,x_1,y_1)*(z_2,x_2,y_2) = (z_1+z_2+y_1(x_2),x_1+x_2,y_1+y_2)$ pour
    tout $(z_1,x_1,y_1),(z_2,x_2,y_2)\in\mathcal{H}(V)$.
\end{definition}

\begin{definition}[Inverse d'un élément d'Heisenberg]
    \label{def:inv_H}
    \uses{def:heisenberg}
    \lean{Heisenberg.inverse}
    \leanok 

    L'inverse d'un élément $(z,x,y)\in\mathcal{H}(V)$ est donné par 
    $(-z- y(-x), - x ,- y)$.
\end{definition}

\begin{proposition}[Groupe d'Heisenberg]
    \label{prop:group_H}
    \uses{def:heisenberg,def:mul_H,def:inv_H}
    \lean{Heisenberg.group}
    \leanok
    Muni de la loi de composition interne, Heisenberg est un groupe.
    \begin{proof}
        \leanok
        On vérifie les axiomes et on fait les calculs. Pas de difficultés
        notoires.
    \end{proof}
\end{proposition}

\section{Centre}

\begin{definition}[Centre d'Heisenberg]
    \label{def:center_H}
    \uses{def:heisenberg}
    \lean{Heisenberg.center}
    \leanok 

    On définit le centre d'Heisenberg comme étant l'ensemble des éléments
    $\mathcal{Z}_{\mathcal{H}(V)}:=\{(z,0,0)\in\mathcal{H}(V),\ z\in k\}$
\end{definition}

\begin{proposition}[Centre est un sous-groupe]
    \label{prop:center_H_subgroup}
    \uses{def:center_H,prop:group_H}
    \lean{Heisenberg.center_is_subgroup}
    \leanok

    Le centre d'Heisenberg $\mathcal{Z}_{\mathcal{H}(V)}$ est un sous-groupe
    de $\mathcal{H}(V)$.
    \begin{proof}
        \leanok
        On vérifie les axiomes et on fait les calculs. Pas de difficultés notoires.
    \end{proof}
\end{proposition}

\begin{proposition}[Caractérisation du centre]
    \label{prop:center_H_is_center}
    \uses{prop:center_H_subgroup}
    \lean{Heisenberg.center_eq}
    \leanok

    Le centre définit plus haut est bien le centre de $\mathcal{H}(V)$.
    \begin{proof}
        \leanok
        Par double inclusion. Le sens réciproque utilise la non dégénérence de la forme
        quadratique \textit{voir plus tard}.
    \end{proof}
\end{proposition}

\section{Commutateurs et nilpotence}

\begin{proposition}[Commutateur]
    \label{prop:commutator_H_of_elements}
    \uses{prop:group_H}
    \lean{Heisenberg.commutator_of_elements}
    \leanok 

    Soient $H_1:=(z_1,x_1,y_1)$ et $H_2:=(z_2,x_2,y_2)$ deux éléments de 
    $\mathcal{H}(V)$. Le commutateur $[H_1,H_2]$ est $(y_1(x_2)-y_2(x_1),0,0)$.
    \begin{proof}
        \leanok
        On calcule $H_1*H_2*H_1^{-1}*H_2^{-1}$.
    \end{proof}
\end{proposition}

\begin{proposition}[Commutateur non trivial]
    \label{prop:commutator_H_nontrivial}
    \uses{prop:center_H_subgroup}
    \lean{Heisenberg.commutator_ne_bot}
    \leanok 

    Si $V$ est non trivial, le sous-groupe de $\mathcal{H}(V)$ engendré par les commutateurs est non trivial.
    \begin{proof}
        \leanok 
        On procède par l'absurde. S'il était trivial, tout élément de $\mathcal{H}(V)$
        appartiendrait à son centre.
        Comme $V$ non trivial, $\exists x\in V,\ x\ne0$. En particulier,
        $(0,x,0)$ y appartiendrait. Mais par définition du centre, $x$ est nul. Contradiction.
    \end{proof}
\end{proposition}

\begin{proposition}[Caractérisation du commutateur]
        \label{prop:commutator_H_caracterisation}
        \uses{prop:commutator_H_of_elements, prop:center_H_subgroup,prop:center_H_is_center}
        \lean{Heisenberg.commutator_caracterisation}
        \leanok
    
        Si $h:=(z,x,y)$ est un élément du commutateur, alors $x=0$ et $y=0$.
    \begin{proof}
        \leanok
        On calcule.
    \end{proof}
\end{proposition}

\begin{theorem}[Nilpotence du groupe d'Heisenberg]
    \label{thm:two_step_nil}
    \uses{prop:commutator_H_nontrivial,prop:commutator_H_caracterisation}
    \lean{Heisenberg.two_step_nilpotent}
    \leanok

    Si $V$ est non trivial, le groupe $\mathcal{H}(V)$ est nilpotent de pas 2.
    \begin{proof}
        \leanok
        Il faut montrer que le commutateur est non trivial et que $[[\mathcal{H}(V),\mathcal{H}(V)],\mathcal{H}(V)]$
        est trivial. Le premier point est fait en \ref{prop:commutator_H_nontrivial}.
        Le second est calculatoire, en utilisant \ref{prop:commutator_H_caracterisation}, pas de difficultés notoires.
    \end{proof}
\end{theorem}

\section{Suite exacte}

\begin{definition}[Morphisme de $k$ vers $\mathcal{H}(V)$]
    \label{def:map_k_H}
    \uses{prop:group_H}
    \lean{Heisenberg.Hom_k_to_H}
    \leanok 

    L'application $\varphi:z\mapsto (z,0,0)$ définie un morphisme de 
    $(k,+)$ dans $\mathcal{H}(V)$.
    \begin{proof}
        \leanok
        C'est clair.
    \end{proof}
\end{definition}

\begin{proposition}[]
    \label{prop:inj_map_k_H}
    \uses{def:map_k_H}
    \lean{Heisenberg.injective_Hom_k_to_H}
    \leanok 

    Le morphisme définit en \ref{def:map_k_H} est injectif.
    \begin{proof}
        \leanok
        On suppose $\varphi(x)=\varphi(y)$ et on montre $x=y$. Pas de 
        difficultés notoires.
    \end{proof}
\end{proposition}

\begin{definition}[Morphisme de $\mathcal{H}(V)$ vers $V\times V^*$]
    \label{def:map_H_VxV*}
    \uses{prop:group_H}
    \lean{Heisenberg.Hom_H_to_V_x_Dual}
    \leanok 

    L'application $\psi:(z,x,y)\mapsto (x,y)$ définie un morphisme de 
    $\mathcal{H}(V)$ dans $V\times V^*$.
    \begin{proof}
        \leanok
        C'est clair.
    \end{proof}
\end{definition}

\begin{proposition}[]
    \label{prop:surj_map_H_VxV*}
    \uses{def:map_H_VxV*}
    \lean{Heisenberg.Hom_H_to_V_x_Dual}
    \leanok 

    Le morphisme définit en \ref{def:map_H_VxV*} est surjectif.
    \begin{proof}
        \leanok
        C'est clair.
    \end{proof}
\end{proposition}

\begin{proposition}[Suite exacte]
    \label{exact_seq_h}
    \uses{def:map_k_H,def:map_H_VxV*}
    \lean{Heisenberg.exact_sequence}
    \leanok 

    On a une suite exacte $0\rightarrow k \rightarrow \mathcal{H}(V) \rightarrow
    V\times V^* \rightarrow 0$.
    \begin{proof}
        \leanok
        C'est clair par ce qui est montrée en \ref{prop:map_k_H} et en \ref{prop:map_H_VxV*}.
    \end{proof}
\end{proposition}
    