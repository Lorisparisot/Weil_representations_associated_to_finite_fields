\section{Construction}

\begin{definition}[Structure of Heisenberg]
    \label{def:heisenberg}
    \lean{Heisenberg}
    \leanok

    Given $k$ a field, $V$ a $k$ vector space and $V^*$ its dual vector space,
    we define the Heisenberg set associated to $V$ by
    $\mathcal{H}(V):=\{(z,x,y) \in k\times V\times V^*\}$.
\end{definition}

\begin{proposition}[Trivial bijection]
    \label{prop:bij_H_triv}
    \uses{def:heisenberg}
    \lean{Heisenberg.bij_k_V_Dual}
    \leanok
    $\mathcal{H}(V)$ is in bijection with $k\times V\times V^*$.
\end{proposition}
\begin{proof}
    \leanok
    Trivial.
\end{proof}

\begin{definition}[Multiplication on Heisenberg]
    \label{def:mul_H}
    \uses{def:heisenberg}
    \lean{Heisenberg.mul}
    \leanok 

    We define an internal law on Heisenberg by the following formula :
    $(z_1,x_1,y_1)*(z_2,x_2,y_2) = (z_1+z_2+y_1(x_2),x_1+x_2,y_1+y_2)$ for every
     $(z_1,x_1,y_1),(z_2,x_2,y_2)\in\mathcal{H}(V)$.
\end{definition}

\begin{definition}[Inverse of an element of Heisenberg]
    \label{def:inv_H}
    \uses{def:heisenberg}
    \lean{Heisenberg.inverse}
    \leanok 

    The inverse of $(z,x,y)\in\mathcal{H}(V)$ is given by the formula
    $(-z- y(-x), - x ,- y)$.
    \begin{proof}
        \leanok 
        Compute $h*h^{-1}$.
    \end{proof}
\end{definition}

\begin{proposition}[Heisenberg's group]
    \label{prop:group_H}
    \uses{def:heisenberg,def:mul_H,def:inv_H}
    \lean{Heisenberg.group}
    \leanok
    Heisenberg is a group for the the internal law defined in \ref{def:mul_H}.
\end{proposition}
\begin{proof}
    \leanok
    We check the axioms of a group.
\end{proof}

\begin{definition}[Bijectivity with $\mathcal{H}(V*)$]
    \label{def:bij_H}
    \uses{def:heisenberg,def:identification_dual}
    \lean{Heisenberg.equiv_Dual}
    \leanok 

    Under our identification of the bidual, the map
    $\Phi : (z,x,y) \mapsto (z,y,x)$ defines a bijection 
    between $\mathcal{H}(V)$ and $\mathcal{H}(V^*)$.
    \begin{proof}
        \leanok
        Compute $\Phi\circ\Phi^{-1}$ and $\Phi^{-1}\circ\Phi$.
    \end{proof}
\end{definition}

\begin{definition}[Antiisomorphic with $\mathcal{H}(V*)$]
    \label{prop:antiisoH}
    \uses{def:bij_H}
    \lean{Heisenberg.anti_iso_Dual}
    \leanok 

    Under our identification of the bidual, the map define in \ref{def:bij_H} is
    a group antiisomorphism from $\mathcal{H}(V)$ to $\mathcal{H}(V^*)$.
    \begin{proof}
        \leanok
        Compute that $\Phi(h_1*h_2)=\Phi(h_2)*\Phi(h_1)$.
    \end{proof}
\end{definition}

\section{Center}

\begin{definition}[Center of Heisenberg]
    \label{def:center_H}
    \uses{def:heisenberg}
    \lean{Heisenberg.center}
    \leanok 

    We define the center of $\mathcal{H}(V)$ by the set
    $\mathcal{Z}_{\mathcal{H}(V)}:=\{(z,0,0)\in\mathcal{H}(V),\ z\in k\}$
\end{definition}

\begin{proposition}[The center is a subgroup]
    \label{prop:center_H_subgroup}
    \uses{def:center_H,prop:group_H}
    \lean{Heisenberg.center_is_subgroup}
    \leanok

    The center of Heisenberg $\mathcal{Z}_{\mathcal{H}(V)}$ is a subgroup
    of $\mathcal{H}(V)$.
\end{proposition}
\begin{proof}
    \leanok
    We check the axioms and compute.
\end{proof}

\begin{proposition}[Caracterisation of the center]
    \label{prop:center_H_is_center}
    \uses{prop:center_H_subgroup,def:non_dege_form_com}
    \lean{Heisenberg.center_eq}
    \leanok

    The set define in \ref{def:center_H} is indeed the center of $\mathcal{H}(V)$.
\end{proposition}
\begin{proof}
    \leanok
    By double inclusion. Reciprocity use the fact that the quadratic form \ref{def:form_com}
    is a non degeneracy one \textit{(see \ref{def:non_dege_form_com} for a proof)}.
\end{proof}

\section{Commutator and nilpotency}

\begin{proposition}[Commutator]
    \label{prop:commutator_H_of_elements}
    \uses{prop:group_H}
    \lean{Heisenberg.commutator_of_elements}
    \leanok 

    Let $H_1:=(z_1,x_1,y_1)$ and $H_2:=(z_2,x_2,y_2)$ be two elements of 
    $\mathcal{H}(V)$. The commutator of $[H_1,H_2]$ is $(y_1(x_2)-y_2(x_1),0,0)$.
\end{proposition}
\begin{proof}
    \leanok
    We compute $H_1*H_2*H_1^{-1}*H_2^{-1}$.
\end{proof}

\begin{proposition}[Commutator isn't trivial]
    \label{prop:commutator_H_nontrivial}
    \uses{prop:center_H_subgroup}
    \lean{Heisenberg.commutator_ne_bot}
    \leanok 

    If $V$ isn't trivial, the subgroup of $\mathcal{H}(V)$ 
    generates by commutators isn't trivial too.
\end{proposition}
\begin{proof}
    \leanok 
    By contradiction. If it was trivial, every element of $\mathcal{H}(V)$
    would belong to its center.
    Because $V$ isn't trivial, there exists $x\in V$ such that $x\ne0$. Thus,
    $(0,x,0)$ would belong to the center. But because of the definition of the center, $x = 0$. 
    We get a contradiction.
\end{proof}

\begin{proposition}[Caracterisation of the commutator]
        \label{prop:commutator_H_caracterisation}
        \uses{prop:commutator_H_of_elements, prop:center_H_subgroup,prop:center_H_is_center}
        \lean{Heisenberg.commutator_caracterisation}
        \leanok
    
        If $h:=(z,x,y)$ belongs to the commutator subgroup, then $x=0$ and $y=0$.
\end{proposition}
\begin{proof}
    \leanok
    We compute.
\end{proof}

\begin{theorem}[Nilpotency of Heisenberg's group]
    \label{thm:two_step_nil}
    \uses{prop:commutator_H_nontrivial,prop:commutator_H_caracterisation}
    \lean{Heisenberg.two_step_nilpotent}
    \leanok

    If $V$ isn't trivial, $\mathcal{H}(V)$ is a two step nilpotent group.
\end{theorem}
\begin{proof}
    \leanok
    We have to show that the commutator isn't trivial and that $[[\mathcal{H}(V),\mathcal{H}(V)],\mathcal{H}(V)]$
    is trivial. The first point is done in \ref{prop:commutator_H_nontrivial}.
    The second is some computation, using \ref{prop:commutator_H_caracterisation}.
\end{proof}

\section{Short exact sequence}

\begin{definition}[Homomorphism from $k$ to $\mathcal{H}(V)$]
    \label{def:map_k_H}
    \uses{prop:group_H}
    \lean{Heisenberg.Hom_k_to_H}
    \leanok 

    The map $\varphi:z\mapsto (z,0,0)$ defines a homomorphism from
    $(k,+)$ to $\mathcal{H}(V)$.
    \begin{proof}
        \leanok
        Trivial.
    \end{proof}
\end{definition}

\begin{proposition}[Injectivity of $\varphi$]
    \label{prop:inj_map_k_H}
    \uses{def:map_k_H}
    \lean{Heisenberg.injective_Hom_k_to_H}
    \leanok 

    The homomorphism defined in \ref{def:map_k_H} is injective.
\end{proposition}
\begin{proof}
    \leanok
    We suppose $\varphi(x)=\varphi(y)$ and we show $x=y$. No difficulties.
\end{proof}

\begin{definition}[Homomorphism from $\mathcal{H}(V)$ to $V\times V^*$]
    \label{def:map_H_VxDual}
    \uses{prop:group_H}
    \lean{Heisenberg.Hom_H_to_V_x_Dual}
    \leanok 

    The map $\psi:(z,x,y)\mapsto (x,y)$ defines a homomorphism from 
    $\mathcal{H}(V)$ to $V\times V^*$.
    \begin{proof}
        \leanok
        Trivial.
    \end{proof}
\end{definition}

\begin{proposition}[Surjectivity of $\psi$]
    \label{prop:surj_map_H_VxDual}
    \uses{def:map_H_VxDual}
    \lean{Heisenberg.surjective_Hom_H_to_V_x_Dual}
    \leanok 

    The homomorphism defined in \ref{def:map_H_VxDual} is surjective.
\end{proposition}
\begin{proof}
    \leanok
    Trivial.
\end{proof}

\begin{proposition}[Short exact sequence]
    \label{exact_seq_h}
    \uses{def:map_k_H,def:map_H_VxDual}
    \lean{Heisenberg.exact_sequence}
    \leanok 

    We have a short exact sequence $0\rightarrow k \stackrel{\varphi}{\rightarrow} \mathcal{H}(V) \stackrel{\psi}{\rightarrow}
    V\times V^* \rightarrow 0$.
\end{proposition}
\begin{proof}
    \leanok
    We check that the kernel of $\psi$ is exactly the image of $\varphi$.
\end{proof}

\begin{definition}[$\psi^{-1}(V)$]
    \label{def:psi_inv_V}
    \uses{def:map_H_VxDual}
    \lean{Heisenberg.Hom_H_to_V_x_Dual_sub_V}
    \leanok 

    The pullback $V\times\{0\}$ by $\psi$ defines a subgroup of $\mathcal{H}(V)$.
    \begin{proof}
        \leanok
        We check that it is a subgroup.
    \end{proof}
\end{definition}

\begin{proposition}[Pullback is commutative]
    \label{prop:psi_inv_V_comm}
    \uses{def:psi_inv_V}
    \lean{Heisenberg.Hom_H_to_V_x_Dual_sub_V_commutative}
    \leanok

    The subgroup defined in \ref{def:psi_inv_V} is commutative.
\end{proposition}
\begin{proof}
    \leanok
    Check that $h_1*h2=h2*h1$.
\end{proof}

\begin{proposition}[Pullback is normal]
    \label{prop:psi_inv_V_normal}
    \uses{def:psi_inv_V}
    \lean{Heisenberg.Hom_H_to_V_x_Dual_sub_V_normal}
    \leanok

    The subgroup defined in \ref{def:psi_inv_V} is a normal subgroup of Heisenberg.
\end{proposition}
\begin{proof}
    \leanok 
    Check that $g*h*g^{-1}\in \psi^{-1}(V)$ for every $g\in \mathcal{H}(V)$ and $h\in \psi^{-1}(V)$.
\end{proof}

\begin{proposition}[Maximality of the pullback]
    \label{prop:psi_inv_V_max}
    \uses{def:psi_inv_V,prop:psi_inv_V_comm}
    \lean{Heisenberg.Hom_H_to_V_x_Dual_sub_V_maximal}
    \leanok

    The subgroup defined in \ref{def:psi_inv_V} is maximal among the
    commutative subgroups of $\mathcal{H}(V)$.
\end{proposition}
\begin{proof}
    \leanok 
    By contradiction. If it's not, then there exists $Q$ a commutative subgroup
    such that $\psi^{-1}(V)\subset Q$ and $Q\ne \psi^{-1}(V)$. Let $q:=(z,x,y)\in Q\backslash \psi^{-1}(V)$.
    In particular, we have for every $h=(a,b,0)\in \psi^{-1}(V)$ that $x*h=h*x$.
    We compute this equality and find out that for every $b\in V$, $y(b)=0$. Thus $y=0$ and $q\in \psi^{-1}(V)$.
    Contradiction.
\end{proof}

\begin{definition}[$\psi^{-1}(V)$]
    \label{def:psi_inv_Dual}
    \uses{def:map_H_VxDual}
    \lean{Heisenberg.Hom_H_to_V_x_Dual_sub_Dual}
    \leanok 

    The pullback of $\{0\}\times V^*$ by $\psi$ defines a subgroup of $\mathcal{H}(V)$.
    \begin{proof}
        \leanok
        We check the axioms.
    \end{proof}
\end{definition}
    
\begin{proposition}[Commutativity of the pullback]
    \label{prop:psi_inv_Dual_comm}
    \uses{def:psi_inv_Dual}
    \lean{Heisenberg.Hom_H_to_V_x_Dual_sub_Dual_commutative}
    \leanok

    The subgroup defined in \ref{def:psi_inv_Dual} is commutative.
\end{proposition}
\begin{proof}
    \leanok
    Check that $h_1*h2=h2*h1$.
\end{proof}

\begin{proposition}[Pullback is normal]
    \label{prop:psi_inv_Dual_normal}
    \uses{def:psi_inv_Dual}
    \lean{Heisenberg.Hom_H_to_V_x_Dual_sub_Dual_normal}
    \leanok

    The subgroup defined in \ref{def:psi_inv_Dual} is ai normal subgroup of Heisenberg. .
\end{proposition}
\begin{proof}
    \leanok 
    Check that $g*h*g^{-1}\in \psi^{-1}(V^*)$ for every $g\in \mathcal{H}(V)$ and $h\in \psi^{-1}(V^*)$.
\end{proof}

\begin{proposition}[Maximality of the pullback]
    \label{prop:psi_inv_Dual_max}
    \uses{def:psi_inv_Dual,prop:psi_inv_Dual_comm}
    \lean{Heisenberg.Hom_H_to_V_x_Dual_sub_Dual_maximal}
    \leanok

    The subgroup defined in \ref{def:psi_inv_Dual} is maximal among the
    commutative subgroups of $\mathcal{H}(V)$.
\end{proposition}
\begin{proof}
    \leanok 
    By contradiction. If it's not, then there exists $Q$ a commutative subgroup
    such that $\psi^{-1}(V^*)\subset Q$ and $Q\ne \psi^{-1}(V^*)$. Let $q:=(z,x,y)\in Q\backslash \psi^{-1}(V^*)$.
    In particular, we have for every $h=(a,0,b)\in \psi^{-1}(V^*)$ that $x*h=h*x$.
    We compute this equality and find out that for every $b\in V^*$, $b(x)=0$. Thus $x=0$ and $q\in \psi^{-1}(V^*)$.
    Contradiction.
\end{proof}

\section{Specifities of the case $k=\mathbb{F}_q$}

\begin{proposition}[Cardinality]
    \label{prop:H_card}
    \uses{prop:bij_H_triv}
    \lean{Heisenberg.card_H}
    \leanok
    If $k$ is a finite field, then $|\mathcal{H}(V)|=|k|\times|V|^2$.
\end{proposition}
\begin{proof}
    \leanok
    With \ref{prop:bij_H_triv} we know $\mathcal{H}(V)\cong k \times V \times V^*$.
    Because $V$ is finite dimensional, $|V|=|V^*|$, thus we get the result.
\end{proof}

\begin{proposition}[Cardinality of the center]
    \label{prop:H_card_center}
    \uses{def:center_H}
    \lean{Heisenberg.card_center}
    \leanok
    If $k$ is a finite field, then $|\mathcal{Z}_{\mathcal{H}(V)}|=|k|$.
\end{proposition}
\begin{proof}
    \leanok
    Trivial, the center being isomorphic to $k$.
\end{proof}

\begin{theorem}[Index of the center]
    \label{thm:H_index_center}
    \uses{prop:H_card,prop:H_card_center}
    \lean{Heisenberg.ord_V}
    \leanok

    If $k$ is a finite field, then $[\mathcal{H}(V):\mathcal{Z}_{\mathcal{H}(V)}]=|V|^2$.
\end{theorem}
\begin{proof}
    \leanok
    We use the fact that $[\mathcal{H}(V):\mathcal{Z}_{\mathcal{H}(V)}]=\frac{|\mathcal{H}(V)|}{|\mathcal{Z}_{\mathcal{H}(V)}|}$.
    Given that $|\mathcal{H}(V)|=|k|\times|V|^2$ (because of \ref{prop:H_card}) and $|\mathcal{Z}_{\mathcal{H}(V)}|=|k|$ (because of \ref{prop:H_card_center}), we get the result.
\end{proof}
