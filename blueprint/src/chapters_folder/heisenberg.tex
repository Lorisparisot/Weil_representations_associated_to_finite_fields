\begin{definition}[Ensemble d'Heisenberg]
    \label{def:heisenberg}
    \lean{Heisenberg}
    %\leanok

    Etant donné $k$ un corps, $V$ un espace vectoriel de dimension finie sur
    $k$ et $V^*$ le dual de $V$, on définit l'ensemble d'Heisenberg par 
    ${(z,x,y) \in k\times V\times V^*}$.
\end{definition}

\begin{definition}[Multiplication sur l'ensemble d'Heisenberg]
    \label{def:mul_H}
    \uses{def:heisenberg}
    %\lean{Heisenberg.mul}
    %\leanok 

    On définit la multiplication sur l'ensemble d'Heisenberg par :
    $(z_1,x_1,y_1)*(z_2,x_2,y_2) = (z_1+z_2+y_1(x_2),x_1+x_2,y_1+y_2)$.
\end{definition}

\begin{definition}[Inverse d'un élément d'Heisenberg]
    \label{def:inv_H}
    \uses{def:heisenberg}
    %\lean{Heisenberg.inverse}
    %\leanok 

    L'inverse d'un élément $(z,x,y)$ de l'ensemble d'Heisenberg est donné par 
    $(-z- y(-x), - x ,- y)$.
\end{definition}

\begin{proposition}[Groupe d'Heisenberg]
    \label{prop:group_H}
    \uses{def:heisenberg,def:mul_H,def:inv_H}
    %\lean{Heisenberg.group}
    %\leanok
    Muni de sa multiplication, Heisenberg est un groupe.
    \begin{proof}
        On vérifie les axiomes, calculs.
    \end{proof}
\end{proposition}