This first chapter introduces some notions that weren't formalize in
mathlib about representation theory. The main goal of this part is to build
the induced representation in a particular case and the Frobenius reciprocity
in a particular case.

\section{Group algebra}

\begin{definition}
    \label{def:trivialkHkG}
    \uses{}
    \lean{kG_kH_Module.Map_KHKG}
    \leanok
    Given $\mathbb{K}$ a field, $G$ a group and $H$ a subgroup of $G$, 
    we have a trivial homomorphism from $\mathbb{K}[H]$ to $\mathbb{K}[G]$.
    \begin{proof}
        \leanok
        Trivial.
    \end{proof}
\end{definition}


\begin{proposition}
    \label{prop:KGcommsemiring}
    \uses{}
    \lean{kG_kH_Module.KGCommRing}
    \leanok
    If $G$ is a commutative group, then $\mathbb{K}[G]$ is a commutative semiring.
\end{proposition}
\begin{proof}
    \leanok
    Trivial.
\end{proof}

\begin{definition}
    \label{prop:KHKGsmul}
    \uses{def:trivialkHkG}
    \lean{kG_kH_Module.SMulKHKG}
    \leanok
    If $H$ is a commutative subgroup of $G$, we have a scalar multiplication
    of $\mathbb{K}[H]$ on $\mathbb{K}[G]$ given by the classic multiplication
    on $\mathbb{K}[G]$.
\end{definition}

\begin{definition}
    \label{def:RingHomkHkG}
    \uses{def:trivialHG,prop:KGcommsemiring}
    \lean{kG_kH_Module.RingMorphism_KH_KG}
    \leanok
    If $H$ is a commutative subgroup of $G$, we have a ring homomorphism
    from $\mathbb{K}[H]$ to $\mathbb{K}[G]$ given by the map defined in \ref{def:trivialHG}.
\end{definition}
\begin{proof}
    \leanok
    Trivial.
\end{proof}

\begin{proposition}
    \label{prop:kG_is_kGcenter_Algebra}
    \uses{def:RingHomkHkG,prop:KHKGsmul}
    \lean{kG_kH_Module.KG_is_KcenterG_Algebra}
    \leanok
    We have that $\mathbb{K}[G]$ is a $\mathbb{K}[\mathcal{Z}_G]$
    algebra, where $\mathcal{Z}_G$ is the center of $G$
\end{proposition}
\begin{proof}
    \leanok
    We check the axioms of an algebra. Nothing really difficult.
\end{proof}

\begin{proposition}
    \label{prop:kG_is_kH_Algebra}
    \uses{prop:kG_is_kGcenter_Algebra}
    \lean{kG_kH_Module.KG_is_KH_Algebra}
    \leanok
    If there exists a homomorphism $H\rightarrow \mathcal{Z}_G$, then $\mathbb{K}[G]$
    is a $\mathbb{K}[H]$ algebra.
\end{proposition}
\begin{proof}
    \leanok
    Using \ref{prop:kG_is_kGcenter_Algebra}, we have \textit{mettre preuve propre}.
\end{proof}

\section{Induced representation}

\begin{definition}
    \label{def:induced_tensor}
    \uses{prop:kG_is_kH_Algebra}
    \lean{Induced_rep_center.tensor}
    \leanok
    Given $G$ a group, $\mathbb{K}$ a field, $W$ a $k$ vector space and $\theta$ a representation of
    $\mathcal{Z}_G$ on $W$, we define the tensor product $V:=\mathbb{K}[G]\otimes_{\mathbb{K}[\mathcal{Z}_G]}V_\theta$,
    where $V_\theta$ is the $\mathbb{K}[\mathcal{Z}_G]$ module associated to $\theta$.
\end{definition}

\begin{proposition}
    \label{prop:induced_add_comm_mono}
    \uses{def:induced_tensor}
    \lean{Induced_rep_center.tensor_add_comm_mon}
    \leanok 
    The $V$ defined in \ref{def:induced_tensor} is an additive commutative monoid.
\end{proposition}
\begin{proof}
    \leanok
    We do \textit{mettre la preuve}.
\end{proof}

\begin{proposition}
    \label{prop:tensor_module_mono}
    \uses{prop:kG_is_kH_Algebra,def:induced_tensor}
    \lean{Induced_rep_center.tensor_module_mono}
    \leanok 
    The $V$ defined in \ref{def:induced_tensor} is a $\mathbb{K}[G]$ module.
\end{proposition}
\begin{proof}
    \leanok
    We do \textit{mettre la preuve}.
\end{proof}

\begin{definition}[Induced representation by the center]
    \label{def:induced_rep}
    \uses{prop:tensor_module_mono,prop:induced_add_comm_mono}
    \lean{Induced_rep_center.definition}
    \leanok
    The $V$ defined in \ref{def:induced_tensor} defined a representation of $G$ called the
    induced representation by $\mathcal{Z}_G$.
\end{definition}

\begin{proposition}
    \label{prop:induced_rep_is_module}
    \uses{def:induced_rep}
    \lean{Induced_rep_center.module_sub_rep_iso}
    \leanok 
   The induced representation defined in \ref{def:induced_rep} is a $\mathbb{K}[\mathcal{Z}_G]$ module.
\end{proposition}
\begin{proof}
    \leanok
    It comes from general properties of tensor products.
\end{proof}

\begin{definition}[Subrepresentation of the induced]
    \label{def:induced_rep_subrep}
    \uses{def:induced_rep}
    \lean{Induced_rep_center.module_sub_rep}
    \leanok
    We define the subrepresentation of \ref{def:induced_rep} by 
    $\mathbb{K}[\mathcal{Z}_G]\otimes_{\mathbb{K}[\mathcal{Z}_G]}V_\theta$,
    where $V_\theta$ is the $\mathbb{K}[\mathcal{Z}_G]$ module associated to $\theta$.
\end{definition}

\begin{proposition}[Coercion]
    \label{prop:induced_rep_coe}
    \uses{def:induced_tensor,def:induced_rep,def:RingHomkHkG}
    \lean{Induced_rep_center.Coe}
    \leanok 
   We have a coercion from element of type $\mathbb{K}[\mathcal{Z}_G]\otimes_{\mathbb{K}[\mathcal{Z}_G]}V_\theta$
   to element of type $\mathbb{K}[G]\otimes_{\mathbb{K}[\mathcal{Z}_G]}V_\theta$.
\end{proposition}
\begin{proof}
    \leanok
    It comes from \ref{def:RingHomkHkG}.
\end{proof}

\begin{proposition}[Isomorphism of subrepresentation]
    \label{prop:induced_rep_sub_iso}
    \uses{def:induced_rep}
    \lean{Induced_rep_center.module_sub_rep_iso}
    \leanok 
   We have a bijection beetween $\mathbb{K}[\mathcal{Z}_G]\otimes_{\mathbb{K}[\mathcal{Z}_G]}V_\theta$
   and $V_\theta$.
\end{proposition}
\begin{proof}
    \leanok
    It comes from a more general theorem which I need to implement.
\end{proof}


