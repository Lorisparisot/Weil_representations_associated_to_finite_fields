This first chapter introduces some notions that weren't formalize in
mathlib about representation theory. The main goal of this part is to build
the induced representation in a particular case and the Frobenius reciprocity
in a particular case.

\section{Group algebra}

\begin{definition}
    \label{def:trivialkHkG}
    \uses{}
    \lean{kG_kH_Module.Map_KHKG}
    \leanok
    Given $\mathbb{K}$ a field, $G$ a group and $H$ a subgroup of $G$, 
    we have a trivial homomorphism from $\mathbb{K}[H]$ to $\mathbb{K}[G]$.
    \begin{proof}
        \leanok
        Trivial.
    \end{proof}
\end{definition}


\begin{proposition}
    \label{prop:KGcommsemiring}
    \uses{}
    \lean{kG_kH_Module.KGCommRing}
    \leanok
    If $G$ is a commutative group, then $\mathbb{K}[G]$ is a commutative semiring.
\end{proposition}
\begin{proof}
    \leanok
    Trivial.
\end{proof}

\begin{definition}
    \label{prop:KHKGsmul}
    \uses{def:trivialkHkG}
    \lean{kG_kH_Module.SMulKHKG}
    \leanok
    If $H$ is a commutative subgroup of $G$, we have a scalar multiplication
    of $\mathbb{K}[H]$ on $\mathbb{K}[G]$ given by the classic multiplication
    on $\mathbb{K}[G]$.
\end{definition}

\begin{definition}
    \label{def:RingHomkHkG}
    \uses{def:trivialkHkG,prop:KGcommsemiring}
    \lean{kG_kH_Module.RingMorphism_KH_KG}
    \leanok
    If $H$ is a commutative subgroup of $G$, we have a ring homomorphism
    from $\mathbb{K}[H]$ to $\mathbb{K}[G]$ given by the map defined in \ref{def:trivialkHkG}.
\end{definition}
\begin{proof}
    \leanok
    Trivial.
\end{proof}

\begin{proposition}
    \label{prop:kG_is_kGcenter_Algebra}
    \uses{def:RingHomkHkG,prop:KHKGsmul}
    \lean{kG_kH_Module.KG_is_KcenterG_Algebra}
    \leanok
    We have that $\mathbb{K}[G]$ is a $\mathbb{K}[\mathcal{Z}_G]$
    algebra, where $\mathcal{Z}_G$ is the center of $G$
\end{proposition}
\begin{proof}
    \leanok
    We check the axioms of an algebra. Nothing really difficult.
\end{proof}

\begin{proposition}
    \label{prop:kG_is_kH_Algebra}
    \uses{prop:kG_is_kGcenter_Algebra}
    \lean{kG_kH_Module.KG_is_KH_Algebra}
    \leanok
    If there exists a homomorphism $H\rightarrow \mathcal{Z}_G$, then $\mathbb{K}[G]$
    is a $\mathbb{K}[H]$ algebra.
\end{proposition}
\begin{proof}
    \leanok
    Using \ref{prop:kG_is_kGcenter_Algebra}, we have \textit{mettre preuve propre}.
\end{proof}

\section{Induced representation}

\begin{definition}[Induced representation as module]
    \label{def:induced_tensor}
    \uses{prop:kG_is_kH_Algebra}
    \lean{Induced_rep_center.tensor}
    \leanok
    Given $G$ a group, $\mathbb{K}$ a field, $W$ a $k$ vector space and $\theta$ a representation of
    $\mathcal{Z}_G$ on $W$, we define the tensor product $V:=\mathbb{K}[G]\otimes_{\mathbb{K}[\mathcal{Z}_G]}V_\theta$,
    where $V_\theta$ is the $\mathbb{K}[\mathcal{Z}_G]$ module associated to $\theta$.
\end{definition}

\begin{proposition}
    \label{prop:induced_add_comm_mono}
    \uses{def:induced_tensor}
    \lean{Induced_rep_center.tensor_add_comm_mon}
    \leanok 
    The $V$ defined in \ref{def:induced_tensor} is an additive commutative monoid.
\end{proposition}
\begin{proof}
    \leanok
    We do \textit{mettre la preuve}.
\end{proof}

\begin{proposition}
    \label{prop:tensor_module_mono}
    \uses{prop:kG_is_kH_Algebra,def:induced_tensor}
    \lean{Induced_rep_center.tensor_module}
    \leanok 
    The $V$ defined in \ref{def:induced_tensor} is a $\mathbb{K}[G]$ module.
\end{proposition}
\begin{proof}
    \leanok
    We do \textit{mettre la preuve}.
\end{proof}

\begin{proposition}
    \label{prop:tensor_module_sub}
    \uses{prop:kG_is_kH_Algebra,def:induced_tensor}
    \lean{Induced_rep_center.tensor_module_sub}
    \leanok 
    The $V$ defined in \ref{def:induced_tensor} is a $\mathbb{K}[\mathcal{Z}_G]$ module.
\end{proposition}
\begin{proof}
    \leanok
    We do \textit{mettre la preuve}.
\end{proof}

\begin{definition}[Induced representation by the center]
    \label{def:induced_rep}
    \uses{prop:tensor_module_mono,prop:induced_add_comm_mono}
    \lean{Induced_rep_center.as_rep}
    \leanok
    The $V$ defined in \ref{def:induced_tensor} defined a representation of $G$ called the
    induced representation by $\mathcal{Z}_G$.
\end{definition}

\begin{definition}[Subrepresentation of the induced]
    \label{def:induced_rep_subrep}
    \uses{def:induced_rep}
    \lean{Induced_rep_center.module_sub_rep}
    \leanok
    We define the subrepresentation of \ref{def:induced_rep} by 
    $\mathbb{K}[\mathcal{Z}_G]\otimes_{\mathbb{K}[\mathcal{Z}_G]}V_\theta$,
    where $V_\theta$ is the $\mathbb{K}[\mathcal{Z}_G]$ module associated to $\theta$.
\end{definition}

\begin{proposition}
    \label{prop:subrep_addmon}
    \uses{def:induced_rep_subrep}
    \lean{Induced_rep_center.module_sub_rep_addcommmon}
    \leanok 
   The tensor product defined in \ref{def:induced_rep_subrep} is an additive commutative monoid.
\end{proposition}
\begin{proof}
    \leanok
    It comes from general properties of tensor products.
\end{proof}

\begin{proposition}
    \label{prop:subrep_module}
    \uses{prop:subrep_addmon}
    \lean{Induced_rep_center.module_sub_rep_addcommmon}
    \leanok 
   The tensor product defined in \ref{def:induced_rep_subrep} is a $\mathbb{K}[\mathcal{Z}_G]$ module.
\end{proposition}
\begin{proof}
    \leanok
    It comes from general properties of tensor products.
\end{proof}


\begin{proposition}
    \label{prop:induced_rep_is_module}
    \uses{def:induced_rep}
    \lean{Induced_rep_center.module_sub_rep_module}
    \leanok 
   The induced representation defined in \ref{def:induced_rep} is a $\mathbb{K}[\mathcal{Z}_G]$ module.
\end{proposition}
\begin{proof}
    \leanok
    It comes from general properties of tensor products.
\end{proof}


\begin{proposition}
    \label{prop:module_sub_rep_iso}
    \uses{def:induced_rep,prop:subrep_module}
    \lean{Induced_rep_center.module_sub_rep_iso}
    \leanok
    We have an isomorphism between $\mathbb{K}[\mathcal{Z}_G]\otimes_{\mathbb{K}[\mathcal{Z}_G]}V_\theta$ and $V_\theta$.
\end{proposition}
\begin{proof}
    It comes from a special case of a theorem \textit{à ajouter}.
\end{proof}

\begin{proposition}[Coercion]
    \label{prop:induced_rep_coe}
    \uses{def:induced_tensor,def:induced_rep,def:RingHomkHkG}
    \lean{Induced_rep_center.Coe}
    \leanok 
   We have a coercion from element of type $\mathbb{K}[\mathcal{Z}_G]\otimes_{\mathbb{K}[\mathcal{Z}_G]}V_\theta$
   to element of type $\mathbb{K}[G]\otimes_{\mathbb{K}[\mathcal{Z}_G]}V_\theta$.
\end{proposition}
\begin{proof}
    \leanok
    It comes from \ref{def:RingHomkHkG}.
\end{proof}

\begin{proposition}[Coercion set]
    \label{prop:induced_rep_coe_set}
    \uses{def:induced_tensor,def:induced_rep,def:RingHomkHkG}
    \lean{Induced_rep_center.Coe}
    \leanok 
   We have a coercion from element of type $Set : \mathbb{K}[\mathcal{Z}_G]\otimes_{\mathbb{K}[\mathcal{Z}_G]}V_\theta$
   to element of type $Set : \mathbb{K}[G]\otimes_{\mathbb{K}[\mathcal{Z}_G]}V_\theta$.
\end{proposition}
\begin{proof}
    \leanok
    It comes from \ref{def:RingHomkHkG}.
\end{proof}

\begin{proposition}[Center as submodule]
    \label{prop:center_submodule}
    \uses{prop:induced_rep_coe_set,def:RingHomkHkG}
    \lean{Induced_rep_center.center_sub_module}
    \leanok 
    The set of elements of $\mathbb{K}[\mathcal{Z}_G]$ seen as elements of $\mathbb{K}[G]$
    defines a $\mathbb{K}[\mathcal{Z}_G]$-submodule of $\mathbb{K}[G]$.
\end{proposition}
\begin{proof}
    \leanok
    We use the coercion to define the set, and then check the axioms.
\end{proof}

\begin{proposition}[$V_\theta$ as submodule]
    \label{prop:theta_submodule}
    \uses{}
    \lean{Induced_rep_center.subrep_sub_module}
    \leanok 
    The set of elements of $V_\theta$ defines a $\mathbb{K}[\mathcal{Z}_G]$-submodule of itself.
\end{proposition}
\begin{proof}
    \leanok
    Trivial.
\end{proof}

\begin{proposition}[$V_\theta$ as submodule isomorphic to $V_\theta$]
    \label{prop:theta_submodule_triv}
    \uses{prop:theta_submodule}
    \lean{Induced_rep_center.subrep_sub_module_iso}
    \leanok 
    The submodule defined in \ref{prop:theta_submodule} is isomorphic to $V_\theta$.
\end{proposition}
\begin{proof}
    \leanok
    Trivial.
\end{proof}

\begin{proposition}[Subrepresentation of the induced one as submodule]
    \label{prop:center_times_theta_submodule}
    \uses{prop:center_submodule,prop:theta_submodule_triv}
    \lean{Induced_rep_center.is_sub_rep_submodule_iso}
    \leanok 
    The image of the map sending \ref{prop:center_submodule} and \ref{prop:theta_submodule_triv}
    to their tensor product defines a $\mathbb{K}[\mathcal{Z}_G]$-submodule of 
    $\mathbb{K}[G]\otimes_{\mathbb{k}[\mathcal{Z}_G]}V_\theta$.
\end{proposition}
\begin{proof}
    \leanok
    Trivial.
\end{proof}

\begin{proposition}[Image of $V_\theta$ as submodule]
    \label{prop:subsubsub}
    \uses{prop:subrep_iso_theta,prop:module_sub_rep_iso}
    \lean{Induced_rep_center.subsubsub}
    \leanok 
    The image of $V_\theta$ by \ref{prop:module_sub_rep_iso} defines a $\mathbb{K}[\mathcal{Z}_G]$-submodule of 
    $\mathbb{K}[G]\otimes_{\mathbb{k}[\mathcal{Z}_G]}V_\theta$, ie $V_\theta$ is a 
    subrepresentation of the induced.
\end{proposition}
\begin{proof}
    \leanok
    Compute the axioms.
\end{proof}

\begin{proposition}[Helpful lemma]
    \label{prop:lemma_tensor}
    \uses{}
    \lean{Induced_rep_center.iso_hom_tens}
    \leanok 
    Given $B$ an $A-algebra$ over a ring $A$, and $M$ a $A-$module and $N$
    a $B-$module, we have an isomorphism $Hom_B(B\otimes_A M, N)\equiv Hom_A(M,N)$. 
\end{proposition}
\begin{proof}
    \leanok
    We use the map $\Psi : \varphi \mapsto \left(x\mapsto \varphi (1\otimes x)\right)$.
\end{proof}

\begin{proposition}[Induced reprensentation property]
    \label{prop:induced_property}
    \uses{prop:lemma_tensor}
    \lean{Induced_rep_center.iso_induced_as_tensor}
    \leanok 
    Let $E$ be a $\mathbb{K}[G]$ module. We have an isomorphism $Hom_{\mathbb{K}[G]}
    \left(\mathbb{K}[G]\otimes_{\mathbb{K}[\mathcal{Z}_G]}V_\theta,\ E\right)\equiv Hom_{\mathbb{K}[\mathcal{Z}_G]}
    \left(\mathbb{K}[\mathcal{Z}_G]\otimes_{\mathbb{K}[\mathcal{Z}_G]}V_\theta,\ E\right)$.
\end{proposition}
\begin{proof}
    \leanok
    We use \ref{prop:lemma_tensor} two times.
\end{proof}


\section{Frobenius reciprocity law}

\subsection{Central function and character of the induced}

\begin{definition}[Central function]
    \label{def:central_fun}
    \uses{}
    \lean{Frobenius_reciprocity.conj_class_fun}
    \leanok 
    Given $G$ a group, a function $f$ over $G$ is called central if it is constant on
    the conjugacy classes of $G$ : $f(g^{-1}xg)=f(x)$ for all $g\in G$ and $x\in G$.
\end{definition}

\begin{definition}[Induced central function]
    \label{def:induced_central_fun}
    \uses{def:central_fun}
    \lean{Frobenius_reciprocity.Ind_conj_class_fun}
    \leanok 
    If $H$ is a subgroup of $G$ a finite group, and if $f$ is central over $H$, we define a central function 
    $f_G$ over $G$ (called the induced central function on $f$) by the formula :
    \begin{equation*}
        f_G(x)=\frac{1}{\text{Card}(H)}\sum\limits_{g\in G\ \wedge\ g^{-1}xg\in H}f(g^{-1}xg)
    \end{equation*}
\begin{proof} 
    \leanok
    We check the axiom by reordering the sum with the bijection $x\mapsto g^{-1}x$.
    Still a problem in the lean proof.
\end{proof}
\end{definition}

\begin{definition}[Character as central function]
    \label{def:char_central_fun}
    \uses{def:central_fun}
    \lean{Frobenius_reciprocity.character_as_conj_class_fun}
    \leanok 
    A character is of course a central function over $G$. We empacked this definition
    in Lean.
    \begin{proof}
        \leanok
        Trivial.
    \end{proof}
\end{definition}


\begin{proposition}[Induced character]
    \label{prop:induced_char}
    \uses{def:char_central_fun,def:induced_central_fun,def:induced_rep}
    \lean{Frobenius_reciprocity.Induced_character_is_character_induced_center}
    \leanok 
    The character of the induced representation by $\mathcal{Z}_G$ over $G$ is the induced 
    central function of the character of $\mathcal{Z}_G$.
\end{proposition}
\begin{proof}
    \textit{Preuve à faire}
\end{proof}